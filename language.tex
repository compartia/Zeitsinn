\documentclass[12pt,a4]{article}
\usepackage{amsmath}
\usepackage{amssymb}
\usepackage{amsfonts}
\usepackage{array}
\usepackage{graphicx}% use this package if an eps figure is included.
\usepackage{mathrsfs}
\usepackage{multirow}
\usepackage{siunitx}
\usepackage{acro}
% \usepackage[
% backend=biber,
% style=alphabetic,
% sorting=ynt
% ]{biblatex}

\setlength\topmargin{-1.1in} \addtolength\textheight{2.1in}
\addtolength{\oddsidemargin}{-0.2in}
\addtolength{\evensidemargin}{-0.1in} \textwidth 5.8in
\newcounter{questioncounter}
\newcounter{equestioncounter}
\setlength\parskip{10pt} \setlength\parindent{0in}
\newcommand{\bea}{\begin{eqnarray*}}
\newcommand{\eea}{\end{eqnarray*}}
\newcommand{\beao}{\begin{eqnarray}}
\newcommand{\eeao}{\end{eqnarray}}
\newcommand{\no}{\noindent}

% \addbibresource{refs.bib}




\DeclareAcronym{llms}{
  short=LLMs,
  long=Large language models
}

\DeclareAcronym{agi}{
  short=AGI,
  long=Artificial general intelligence
}

\begin{document}
\title{The Confluence of Inherited and Learned Morality}

\author{Artem K. Zaborskiy }


\maketitle

\begin{abstract}
This essay is an exploration of the interplay between language, ethics, and societies. It delves into how language reflects and shapes our observer-dependent realities, the role of \ac{llms} in ethics and linguistics development, the ethics of speech acts, and the influence of language in the political sphere. The essay aims to unravel the complex relationships between inherited morality and learned linguistic structures, prompting a reevaluation of traditional notions of morality and truth in the light of \ac{agi} development, Exploring the pluralistic nature of ethics without trying to define what is moral and what is not. 
\end{abstract}

\tableofcontents
\printacronyms


\section{Introduction}

Albert Schweitzer's concept of "Ehrfurcht vor dem Leben," or "Reverence for Life," calls for profound respect towards all living beings. Schweitzer regarded life as a precious and enigmatic gift \cite{1}. Although every creature possesses a will to live—a result of naturally selected traits beneficial for survival—this is juxtaposed against the impartiality of evolutionary mechanisms.
\par 
In the field of ethology, substantial insights have been garnered regarding the biological and evolutionary underpinnings of human behavior. However, it is important to exercise caution in directly correlating evolutionary biology with ethical constructs. Notwithstanding this caution, we may observe that certain behaviors conducive to survival, which are encoded within our genetic makeup, exhibit parallelism with what contemporary society delineates as moral principles.

\par Take, for instance, the widely held moral stance that killing members of one's own kind is wrong—a view that in most cases aligns with the evolutionary disadvantage of such behavior — this instinct has perhaps influenced religious doctrines and legislative systems. 
(However, the question remains complex: who exactly is considered 'one's own' kind?)
\par
When inquiring into the rationale behind a widely acknowledged moral principle or law, an analysis often reveals that such a principle is advantageous from an evolutionary or survival perspective. This observation suggests a correlation between commonly accepted ethical norms and their potential benefits in terms of species preservation and evolutionary success.

\par
Changes in the genetic structure demonstrate profound inertia, evolving slowly, especially when compared to the rapid dynamics of modern civilization and language development. It's plausible to suggest that our primal survival instincts have been refined by what we now term as morality.



\section{Language timeline}

Homo sapiens, our own species, has existed for an estimated 200-300 thousand years [source?]. Drawing from archaeological evidence and the anatomical analysis of the human throat (hyoid bone), it is posited that complex linguistic abilities might have been present in early members of the Homo genus, potentially in species such as Homo erectus or Homo heidelbergensis [source?]. On the timeline of human evolution, formal literacy is a relatively recent phenomenon, having emerged around 5,200 years ago with the advent of Sumerian cuneiform [source?].


 


 

\section{System of ideas}
Let us bring a list of several definitions of one term. The quest to define that term has long been a not easy task for philosophers, one that has seen numerous intellectual attempts falter against its complexity. 
While reading, try to guess, what all they refer to:  

\begin{description}
    \item[-] \textit{Emile Durkheim} regarded it as a system of beliefs and practices tied to the sacred, unifying adherents into a moral community.
    \item[-] \textit{Rudolf Otto} spoke of the numinous, an encounter with the holy that is both fascinating and terrifying.
    \item[-] \textit{Paul Tillich }identified it with the ultimate concern, a profound connection with absolute reality.
    \item[-] \textit{Friedrich Schleiermacher} described it as a sense of absolute dependence, a communion with the universe.
    \item[-] \textit{Claude Lévi-Strauss} viewed it as part of a broader system of symbols intrinsic to human culture and cognition.
    \item[-] \textit{Karl Marx} famously criticized it as a palliative illusion.
\end{description}    

But what were they trying to define?\\
Religion.
 \par

The intrinsic diversity of religious experience, which resists a singular or universal description, challenges the very utility of the term. The sacred is multifaceted manifold, and the spiritual experiences it engenders are deeply personal, ineffable and beyond simple description.
\par
In light of these varied perspectives, we may argue that the term 'religion' has indeed outlived its usefulness. We should instead adopt \textit{'system of ideas'} as a more inclusive term that captures the full range of phenomena traditionally labeled as religious. This term would encompass not only traditional religions but also secular ideologies with religious-like features, such as communism or corporate ethics.

\par
Advocating for the term \textit{'system of ideas'} introduces universality and flexibility, permitting its application across a variety of cultural constructs and encouraging an objective, analytical conversation free from the dogmas, biases, and historical baggage associated with traditional religions. This replacement term provides a broader and more adaptable framework for understanding the diverse ways in which humans seek meaning, structure, and transcendence, acknowledging the rich legacy of 'religion' while proposing a dynamic and inclusive alternative.

\par
Earlier we have demonstrated that altering language changes the reality we inhabit. The term 'religion,' stemming from the Latin 'religare,' means 'to bind together' or 'to re-link.' Let us reconsider and reweave the concept of 'religion.'
\par
The concept of 'religion' is often perceived as static—a set framework to which nothing can be casually added or removed. In contrast, a \textit{'System of Ideas'} implies dynamism, inviting the addition of new ideas and the removal of those that have become obsolete.
\par



\subsection{Short overview of modern ethics}

\begin{table}[h!]
    \centering
    \begin{tabular}{p{3cm} p{5cm} p{4cm} } 
    \hline 
 
    \textbf{Utilitarianism:} 
        & Actions are right if they result in the greatest happiness for the most people. 
        & Jeremy Bentham, John Stuart Mill \\ \hline

    \textbf{Deontological Ethics} 
        & Morality is based on rules or duties, irrespective of outcomes.
        & Immanuel Kant \\ \hline
    
    \textbf{Virtue Ethics} 
        & Focuses on the development of virtuous character traits. 
        & Alasdair MacIntyre, Elizabeth Anscombe (drawing from Aristotle)\\ \hline 
    
    \textbf{Existentialism and Morality} 
        & Highlights individual freedom, choice, and responsibility.
        & Jean-Paul Sartre\\ \hline 
        
    \textbf{Neuroethics and Evolutionary Ethics}
        & Studies the neurological and evolutionary underpinnings of moral cognition 
        & Patricia Churchland, Joshua Greene\\ \hline 
         
    \textbf{Contractualism, Contractarianism} 
        & Moral norms arise from a social contract that rational individuals
          would agree to.
        & John Rawls, T.M. Scanlon \\ \hline

    \textbf{Care Ethics} 
        & Morality is rooted in relationships and care for others 
        & Carol Gilligan, Nel Noddings \\ \hline

    \textbf{Moral Realism and Anti-realism }
        & Debate over whether moral propositions are objective (realism) 
        or subjective (anti-realism) 
        & G.E. Moore (realist), J.L. Mackie (anti-realist) \\  \hline
        
    \textbf{Postmodern Ethics} 
        & Questions universal moral frameworks and emphasizes plurality and context 
        & Michel Foucault, Jacques Derrida \\ \hline
        
    \textbf{Neuroethics and Evolutionary Ethics} 
        & Studies the neurological and evolutionary underpinnings of moral cognition 
        & Patricia Churchland, Joshua Greene \\ 
    \hline

    \end{tabular}
    
\end{table}

\section{Reinforcement}
Kant's "moral law within" alone cannot tether our darker impulses. External rewards and deterrents play a role.
\par
\subsubsection{The Reality of the Virtual and the Religious Guilt}
In Slavoj Žižek's exploration of Lacanian psychoanalysis, he introduces the concept of "the reality of the virtual".
In this framework, the potency of punishment derives from its indeterminate nature. It is the uncertainty, the not-knowing, that gives rise to the fear of punishment. 
\par
If a generalized eschatological infernal realm were meticulously described, it might transition from an anxiety-inducing abstraction to a tamed reality, clearly defined and consequently less formidable. It is the \textit{nebulous}, the undefined, that truly haunts us and ought to remain amorphous to preserve its capacity to instill dread. These threats linger best in the space of the virtual, potent and chilling in their ambiguity. In this vein, one might jest that Dante, in his vivid portrayal of the Inferno, may have unwittingly pilfered from us the more terrifying specters of the unknown; he gave us a map where perhaps we were better off with uncertainty.


\section{Experimental procedure}


Fig. $\ref{fig1}$ shows the experimental setup. Four K-type thermocouples were clenched equidistantly to a rod of copper of length about $0.5$ m and diameter $30$ mm. The metallic rod was heated by a square pulse using a $25$ W cartridge heater at a rate of $5$ mHz. The heater was connected to a switching circuit which was controlled by using a Labview program which sends a square pulse to the relay.

\begin{figure}[!h]
\begin{center}
\includegraphics[scale=0.4]{SchemeticDiagram.eps}\\
\caption{Schematic diagram of the experimental setup. } \label{fig1}
\end{center}
\end{figure}
The thermocouples are first calibrated using Stein-Hart Calibration. DAQ card is attached to collect the data from thermocouple and plot the predicted temperature values as a function of time.The process of heating was continued until the dynamic equilibrium had been achieved after the initiation of the setup. \\ Once the dynamic equilibrium had been achieved, Fast Fourier Transform (FFT) was performed on the finely sampled numerical data sets and then was plotted. The odd harmonics were seen by FFT graphs.
With the help of Fourier Transformed graph, the amplitudes of temperature oscillations (in a dynamic equilibrium) of the first thermocouple (TC$1$) and the fourth thermocouple (TC$4$) were measured. With the help of those, the damping coefficient and the velocity of thermal wave  was calculated.
Once the damping coefficient and velocity of the thermal wave had been calculated, the thermal diffusivity ``$D$" was calculated by $D=v/2\epsilon$. Here ``$\epsilon$" represents the damping coefficient.


\section{Results}
Fig. $\ref{fig2}$ shows that the amplitude of the oscillations decreases with the distance far from the origin. This also illustrates that these oscillations are not in phase; there is a phase lag between successive thermocouples. The triangular shape arises out of the choice of actuation frequency and the distance of the first thermocouple from the heater surface. At first thermocouple, there is not fluctuation in the frequency (maximum amplitude) that is why there occurs triangular variation. At the thermocouple which is farthest from the heat source, the temperature fluctuation (much smaller in amplitude) is nearly a perfect sinusoidal.
\begin{figure}[!h]
\begin{center}
\includegraphics[scale=0.4]{AllTCs.eps}\\
\caption{Temperature oscillations at different points along the copper bar. The thermocouples which are nearer to the heat source have higher average temperatures. } \label{fig2}
\end{center}
\end{figure}

Now by looking at the Fig. $\ref{fig3}$ the thermocouples which are closer to the heat source have the larger amplitudes as compared to the thermocouples which are farther away from the heater. It also shows that there occur only odd harmonics. There are only three odd harmonics which decay exponentially.
\begin{figure}[!ht]
\begin{center}
\includegraphics[scale=0.4]{AllFFts.eps}\\
\caption{Fourier transforms of the temperatures measured by the thermocouples. } \label{fig3}
\end{center}
\end{figure}
\newpage
In order to calculate the diffusivity of the material, damping coefficient $\epsilon$ has to be calculated by using following expression
\begin{equation}
\epsilon=\frac{1}{\Delta x}\ln(\frac{A_1}{A_2}) \, , \label{eq6}
\end{equation} 
where $A_1$ and $A_2$ are the respective amplitudes of the first and the fourth thermocouple's oscillations and $\Delta x$ represents the separation distance between them which is $0.06$ m in this experiment. The values of the amplitudes and the phase lag has been taken from the Fig. $\ref{fig4}$ and tabulated in the Table $\ref{tb1}$ and Table $\ref{tb2}$ respectively.

\begin{table}[!h]
\caption{Amplitudes of the oscillations of $1$st and $4$th thermocouple.} \label{tb1}

\centering

\begin{tabular}{ |c|c|c|c| } 
 \hline
   & Lower amplitude (L$_1$) & Higher amplitude (L$_2$) & Difference (L$_2$-L$_1$)\\ 
 \hline
 A$_1$ & 49.81 & 54.77 & 4.96\\ 
 \hline
 A$_2$ & 48.93 & 51.92 & 2.99\\
 \hline

\end{tabular}

\end{table}
\begin{table}[!h]
\caption{Phase lag between $1$st and the $4$th thermocouple.} \label{tb2}

\centering

\begin{tabular}{ |c|c|c|c| } 
 \hline
   $1$st thermocouple phase (t$_1$) & $4$th thermocouple phase (t$_2$) & $\Delta t$(s)\\ 
 \hline
  6034 & 6042 & 8\\ 
 
 \hline
\end{tabular}
\end{table}
Now from the Eq. $\ref{eq6}$, the damping coefficient ($\epsilon$) is $9.235$ m$^{-1}$. With the help of phase lag ($\Delta t$), the wave velocity has been calculated by $v=\Delta x/\Delta t$ and that is $0.0075$ ms$^{-1}$.

Hence, the thermal diffusivity is calculated by following expression
\begin{equation}
D=\frac{v}{2\epsilon}= \SI{4.061e-4} {\meter^2\second^{-1}} 
\end{equation}


\begin{figure}[!ht]
\begin{center}
\includegraphics[scale=0.4]{F3.eps}\\
\caption{Fourier transforms of the temperatures measured by the thermocouples. } \label{fig4}
\end{center}
\end{figure}

\section{Conclusion}

 

The purpose of this experiment was to measure the thermal diffusivity of the copper metal. Hence that purpose is achieved and the experimental value is close enough to the theoretical value \cite{4}. This experiment provides an opportunity to get acquainted with heat conduction in a way that is essentially different from that of classical experiments on stationary heat transmission. This experiment also allows one to learn thermal diffusivity measuring techniques in a simple and pedagogical way.

\begin{thebibliography}{99}

 
 

\bibitem{1} Albert Schweitzer, Bähr, Hans Walte ``\emph{Die Ehrfurcht vor dem Leben: Grundtexte aus fünf Jahrzehnten}", Beck, 9783406494482 (2003).

\bibitem{2} A. Bodas, V. Gandia, E. Lopez-Baeza, ``\emph{An undergraduate experiment on the propagation of thermal waves}", Am. J. Phys. \textbf{66}, 528-1-533 (1998).

\bibitem{3} L. Verdini and A. Santucci, ``\emph{Propagation properties of thermal waves and
thermal diffusivity in metals}", Nuovo Cimento B \textbf{62}, 399-421 (1981).


\bibitem{4} M. S. Anwar, J. Alam, M. Wasif, R. Ullah, S. Shamim, W. Zia, ``\emph{Fourier analysis of thermal diffusive waves}", J. Phy \textbf{82}, 928 (2014).

\bibitem{5} George B. Arfken, Hans J. Weber, ``\emph{Mathematical methods for physicists}", 6th Ed, Chapter 14.

\bibitem{6} K. Etori, ``\emph{Remarks on the temperature propagation and the thermal diffusivity of a solid}", Jpn. J. Appl. Phys. \textbf{11}, 955-957 (1972).

%\bibitem{5} A. Mandelis, ``\emph{Diffusion waves and their uses}", Phys. Today \textbf{66}, 29-34 (2000).

\end{thebibliography}

% \printbibliography

\end{document} 