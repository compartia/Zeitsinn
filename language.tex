% https://www.overleaf.com/project/657c60bf2ef1fbba865e8e75
\documentclass[11pt,a4]{article}

\usepackage{amsmath}
\usepackage{amssymb}
\usepackage{amsfonts}
\usepackage{array}
\usepackage{mathrsfs}
\usepackage{multirow}
\usepackage{siunitx}
\usepackage{acro}
\usepackage[acronym]{glossaries}
\usepackage{hyperref}

\hypersetup{
    colorlinks=true,
    linkcolor=blue,
    filecolor=magenta,
    urlcolor=cyan,
    pdftitle={Overleaf Example},
    pdfpagemode=FullScreen,
    }


\usepackage{booktabs}
\usepackage[LGR, T1]{fontenc}
\usepackage[utf8]{inputenc}   % utf8 is required


\newcounter{questioncounter}
\newcounter{equestioncounter}

\newcommand{\textgreek}[1]{\begingroup\fontencoding{LGR}\selectfont#1\endgroup}

\let\oldmarginpar\marginpar
\renewcommand\marginpar[1]{\-\oldmarginpar[\raggedleft\footnotesize #1]%
{\raggedright\footnotesize #1}}




\newacronym{nn}{NN}{neural network}
\newacronym{llm}{LLM}{large language model}
\newacronym{ai}{AI}{artificial intelligence}
\newacronym{agi}{AGI}{artificial general intelligence}
\newacronym{ave}{AVE}{artificial virtual employee}

\makeglossaries
\begin{document}
\title{The Confluence of Ethics, Language and AI}

\author{Artem K. Zaborskiy\\ \href{mailto:zaborskiy@pm.me}{zaborskiy@pm.me}}

\maketitle

% https://www.overleaf.com/project/657c60bf2ef1fbba865e8e75


\begin{abstract}

This essay is an exploration of the interplay between language, ethics,
and societies. It delves into how language reflects and shapes our
observer-dependent realities, the role of \glspl{llm} in ethics and
linguistics development, the ethics of speech acts, and the influence
of language in the political sphere. The essay aims to unravel the
complex relationships between inherited morality and learned linguistic
structures, prompting a reevaluation of traditional notions of
morality and truth in the light of \gls{ai} development,
exploring the pluralistic nature of ethics without trying
to define what is moral and what is not.
\end{abstract}
% \rule{8cm}{0.4pt}
\par
\begin{center}

\begin{tabular}{r|l}
    \hline
     Version & 1.0  \\
     Status & public draft  \\
     License & MIT  \\
     Source & \href{https://github.com/compartia/Zeitsinn/blob/master/language.tex}{GitHub} \\
     En, Ru versions & \href{https://drive.google.com/drive/folders/1hBFRGCIHgeUH0l3Xg6-QnEXOA5KdAitQ}{Google Docs folder} \\
     \hline
\end{tabular}
\end{center}

\par
\newpage

\tableofcontents


\printglossaries



%
\newpage
\section{Introduction}

    Albert Schweitzer described life as a precious and mysterious
    gift\cite{Schweitzer}. His concept of \textit{“Ehrfurcht vor dem Leben,”}
    or "Reverence for Life," advocates a profound respect for all living beings.
    Yes, every creature is endowed with a will to live, a product of
    natural selection, but evolutionary mechanisms are indifferent---every species eventually becomes
    extinct, making way for new ones.

    The lifespan of an individual, or even a whole species, is
    laughably short compared to the known history of the universe.
    It seems logical, then, to imbue this brief existence with
    meaning and purpose within the vast, almost infinite
    evolution of the universe.

    Perhaps it is this contradiction, this very tension
    between the poles of guaranteed mortality and a
    strong will to live, that grounds the pursuit of
    purpose, meaning, and ways of living. This search
    is solidified in ethics and morality.

    If life is merely a tiny link in the long chain of
    existence, what exactly do we take from the past
    and carry into the future?

    \begin{quote}
        \textit{What criteria
        do we apply to discern what is worth leaving in the past?}
    \end{quote}

\newpage
\section{Cultural and Genetic Dynamics, Foundations of Ethics}

    Let's approach dogmas and well-known moral principles not as
    immutable, eternal axioms bestowed from above, but as constructs
    with discernible causes, aiming to uncover their origins rather than
    justifying them.

    Investigating the foundations of any moral principle or law often
    reveals that such principles are evolutionarily and survivalistically
    advantageous. This suggests a correlation between ethical
    norms and their evolutionary benefits. It seems plausible
    that behaviors beneficial for
    survival initially become ingrained in our genes, manifesting as
    instincts, and are subsequently refined through cultural superstructures.
    These behaviors are then articulated in language,
    codified in legislation, echoed in sacred texts, legends,
    instructive fairy tales, and even moralistic narratives in
    Disney animations, etc.

    \par
    In essence, human language and DNA share a common goal: the \textit{transmission} of
    information across \textit{space and time}\footnote{\\- A typical Bible text file,
    without special formatting, takes about 4-5 MB. \\
    - Human DNA is often quantified in terms of base pairs.
    The human genome contains about 3 billion base pairs.
    The storage capacity of DNA is usually estimated based on the
    fact that each base pair can contain
    2 bits of information (since there are four possible nucleotides: A, T, C, G).
    Thus, the total amount of information that can be stored in the human
    genome is about 750 MB, which corresponds to approximately 150–200 copies of
    the Bible.\\
    - A speculative and unscientific estimate of the capacity
    of the human brain is tens of terabytes, which is
    equivalent to several million books.}.
    Both mechanisms are fundamental to inheritance and
    the sustenance of life, making \textit{this parallel noteworthy.}


    The most critical aspect to emphasize is the \textit{relative speed} of
    development between the two mechanisms. Genetic structure evolves extremely
    slowly compared to the rapid pace of advancements in knowledge transmission
    technologies, particularly language. Consequently, instincts change at a
    sluggish rate, whereas the conditions, environment, and context in
    which our animal instincts still operate are transforming with a
    completely different dynamic.
\subsection{Means of Knowledge Transfer on a Timeline}


    To illustrate this dynamics with numbers , the
    key historical epochs can be briefly summarized:

        \begin{itemize}
            \item[-] Drawing from archaeological evidence and the anatomical analysis of the human hyoid bone,
            it is posited that complex linguistic abilities might have been present in early members of the
            Homo genus, potentially in species such as Homo erectus or Homo heidelbergensis \cite{Capasso2008AHE}.
            \item[-] Homo sapiens, our own species, has existed for an estimated 200-300 thousand
            years\cite{Vidal}.
            \item[-] On the timeline of human evolution, formal literacy is a relatively recent
            phenomenon, having emerged around 5,300 years ago with the advent of Sumerian cuneiform\cite{Walker}.
            \item[-] The Internet was adopted around 50 years ago.
            \item[-] LLMs are 1-2 years old
        \end{itemize}

        \textit{200,000 – 5,000 – 50 – 2:}
        oversimplifying the complex, non-linear nature of human progress and technological progress, one may
        notice that each subsequent era is on average tens of times shorter than the previous one.


\subsection{Language as an Organism. Staticness \& Dynamics}

    The dynamics of survival among various systems of ideas and
    languages can be viewed as akin to the
    evolution of biological species. Metaphorically speaking,
    different nations exist in symbiosis with
    their culture, competing with other nations and their ideologies for resources.
    The primary resource has always been land. Culture, language, mentality, and moral norms are largely
    determined by the characteristics of the landscape and the type of nature.

    The development of language mirrors the evolution of a
    living system, or organism. For the evolution
    of any complex system, maintaining homeostasis
    requires two essential components or tendencies:
    a certain degree of staticness (a) and inertia, alongside dynamics (b), as discussed by Norbert
    Wiener in his book "Cybernetics: Or Control and Communication in the Animal and the Machine"\cite{Wiener}.

    The evolution of human language exemplifies the interplay between these components---staticness and dynamics.
    Natural linguistic changes and mutations are moderated by the process of standardization.

    Postmodernism has increased the dimensionality of the system, or, in other words, increased the
    number of possible states that the system can assume. Trusted developmental mechanisms cannot
    be expected to continue functioning in the same predictable manner, without turbulence, as the
    dimension of space increases. (See also Kernel Trick \ref{kerneltrick})

\subsection{The Battle of Dialects. Norms}

    As the population increased, the complexity of managing public order also grew,
    necessitating clear and unambiguous legislation. This need catalyzed a shift towards
     linguistic unification. The precise formulation of legal terminology became crucial
     in ensuring a uniform interpretation and application of laws in various regions.

    As a result, the dialect used for official purposes, particularly in legal documents
    and procedures, often gained significant prestige. It evolved into the
    standard---a "prestigious" dialect\footnote{It is crucial to recognize that a
    language's high style or a dialect distinct from "rural" or "provincial" ones,
    often featured in media, is not inherently superior to other regional dialects.
    Its prominence is largely a result of historical and sociopolitical factors rather
    than linguistic merit. The dialect in which laws are written, perceived as 'standard'
    or prestigious, gains its status more from its association with power and authority
    than from its intrinsic linguistic qualities.}, establishing a template for official
    communication and embedding itself as the norm\footnote{As a close example,
    Cicero is considered the founder of the normative literary Latin language, and the
    language of all authors who wrote before Cicero's emergence in the literary
    scene (around 80 BC) was regarded as archaic.
    }.


\subsection{Ambiguity versus Specificity}

    In language development, we observe a tension between the increase in ambiguity and the rise of
    specificity---some aspects crystallize while others dissolve. Discussing language in the context
    of morality is vital to ensure that the doctrines represented in narratives are correctly understood
    in the modern context, a concern particularly pertinent in the realm of \acrshort{ai} development.

    Recent advancements in \glspl{llm} can act as linguistic anchors.
    Utilizing computational speed, \glspl{llm}  can track language changes or even resolve discrepancies
    in definitions, offering an "updated dictionary" that reflects unified interpretations of terms.
     While \glspl{llm}  can enhance mutual understanding and provide a basis for consensus,
     it's important to recognize their potential for homogenizing the conceptual framework.
     This linguistic globalization, from an evolutionary perspective, may not be entirely beneficial,
     but to judge evolution would be overly anthropocentric.


 \subsection{Understanding of Misunderstanding.}

        \subsubsection{Shared Matrix through Generations}
            \par
            Historically, life unfolded through predictable patterns. A son's path often echoed his father's;
            communities upheld uniformity. Within this regularity, the stories they lived by were nearly similar.
            Comprehension thrived on a shared matrix of semantic reference points, fostering a
            consistent flow of knowledge.

        \subsubsection{Linguistic Expansion and Ambiguity}

            In today's world, brimming with diverse narratives, linguistic variety, and a stream of digital imagery,
            including AI-generated media, the purity of terms is called into question. One person's understanding,
            shaped by cinema, literature, and personal experience, forms a collage distinct from another
            person's collage, assembled under completely different influences.

            As language expands into new domains, its precision fades. With each new context, a word's meaning
            multiplies, becomes diluted, enhancing its \textit{ambiguity}. Terms are often transposed from old
            contexts into new ones, shaped by emerging technologies.

            For instance,  the initial meanings of "kosmos" (\textgreek{κόσμος}) in ancient Greek culture
            encompassed concepts of order, arrangement, and adornment (cosmetic). Over time, these meanings
            evolved through philosophical discourse to the modern understanding of "cosmos" as the universe
            seen as a well-ordered, harmonious whole.


             Some philosophical problems arise when terms are transferred from an old context to a new context
             created by some new technology. This can potentially lead to the emergence of pseudoproblems and
             pseudoparadoxes, which can often be resolved through clear redefinition of terms. People may
             engage in long and meaningless debates about concepts like god, beauty, truth, faith, fairness,
             goodness, and similar concepts, but the nature of these discussions and disputes often lies
             only in the lack of common definitions of terms, especially when it comes to abstract entities.

        \subsubsection{Presumption of Comprehension}
            Despite the widening gaps in understanding, there persists
            \textit{an unspoken belief} in our ability to understand each other.

            Consider the statement, "I believe in God." Delving into its meaning
            reveals that the speaker's conception of "God" and "belief" is a unique patchwork, weaved from
            various threads of personal experiences
            (which echoes Derrida's view that meaning is not
            absolute but fluid and open to
            various \textit{interpretations} \cite{Deconstruction}).

            "What does 'believe' mean?"---ask any friend, and you will find
            yourself in a fog of vagueness or even in dispute. On the other hand,
            people are more willing to cooperate if they assume (often unconsciously)
            that they have more in common than not. \marginpar{\raggedright What is discussed here has
            certain parallels with the False Consensus Effect\cite{FalseConsensus}.}



            It's not merely that we pretend to grasp each other's meanings; we seldom even acknowledge
            the prospect of \textit{being misunderstood}.

             \begin{quote}
            "We do not understand that we are misunderstood—not a paraphrase, but a hint at rhetorical
            kinship with Socratic \textit{"I know that I know nothing."}\footnote{For entertainment,
            these expressions can be written in terms of Dynamic Epistemic Logic (DEL):

            "We (A) do not understand (U) that [they] (B) misunderstand us":
            \[\neg (K(A, U) \land K(B, \neg U))\]

            "I (A) know (K) that I know nothing":
            \[K(A, \neg K(A, \phi))\]
            }
            \end{quote}

            Is said indeed categorically, but it is to highlight changing the degree
            of misunderstanding/understanding. Nowadays, we need to communicate more with different people,
            whether it's online or in person. We deliberately bring no definition of the concept
            of 'understanding' here, it has been long and extensively
            discussed by philosophers.



        % \begin{quote}
        %     We do not understand that we are misunderstood.
        % \end{quote}

        % Is said indeed categorically, but it is to highlight changing the degree
        % of misunderstanding/understanding. Nowadays, we need to communicate more with different people,
        % whether it's online or in person. We deliberately bring no definition of the concept
        % of 'understanding' here, it has been long and extensively
        % discussed by philosophers.




        We consider understanding in a broader context of interactions
        that goes beyond communication between two conscious individuals. It includes interactions
        between a person and text, media, AI,
        and even AI with AI. In such scenarios
        without long context, we either (1) define common concepts and semantic
        base (e.g., do you speak English?) or (2) assume a
        shared conceptual framework.
        \begin{quote}
            This diversity of interpretation is not a problem to be solved
            but rather a characteristic of language and modernity that
            should be \textit{acknowledged}.
        \end{quote}


        Misunderstandings, or partial understandings, may play an
        unrecognized role in human interaction.
        (About this - further, in the chapter \ref{reality} Reality of the Virtual.)

        The search for common ground is crucial for
        effective cooperation, with the assumed mutual understanding
        serving as a social adhesive - people tend to cooperate more readily
        when they believe they share more similarities than differences.



% ===================


\section{Virtual Punishments \& Rewards.}

% Stimuli and Consequences: Application of Morality:
Kant's mere \textit{'moral law within'} alone cannot suppress our
dark impulses. External rewards and inhibiting factors play a role.
On the other hand,  any moralizing and attempts to impose morality
lead to undesirable side effects. There cannot be
a single \textit{'source of morality'}.
At least, morality and ethics are not something that a
secular state can try to regulate.

When discussing the concepts of truth and falsehood, we cannot
ignore the notions of 'real' and 'illusory'.


 \subsection{The Reality of the Virtual.} \label{reality}

        In Slavoj Žižek's exploration of Lacanian psychoanalysis, he introduces
        the concept of "\textit{the reality of the virtual}" which hinges on the interplay between
        Lacan's three realms of human reality\footnote{"The Unholy Trinity." The \textit{Symbolic}
        encompasses language and laws structuring our social world; The \textit{Imaginary} includes
        our idealized images and perceptions; The \textit{Real} is what resists
        symbolization altogether.}: \textit{the Symbolic, the Imaginary, and the Real}\cite{lacan1981four}.
        In this framework, the potency of punishment derives from its indeterminate
        nature \cite{Borretzen2012-tx}. It is the uncertainty, the not-knowing, that gives rise
        to the fear of punishment.

        If a generalized eschatological infernal realm were meticulously described, it might transition
        from an anxiety-inducing abstraction to a tamed reality, clearly defined and consequently less
        formidable. It is the \textit{nebulous}, the undefined, that truly haunts us and ought to remain
        amorphous to preserve its capacity to instill dread. These threats linger best in the space of
        the virtual, potent and chilling in their ambiguity. In this vein, one might jest that Dante,
        in his vivid portrayal of the Inferno, may have unwittingly pilfered from us the more terrifying
        specters of the unknown; he gave us a map where perhaps we were better off with uncertainty.


    \subsection{Ideological Systems and Spiritual Culpability} \label{religion}

    Although some behavioral aspects have a genetically established basis, others are dictated by
    cultural and social religious experience.

    The interaction of genetic predispositions with cultural/ideological teachings in shaping human
    behavior is the subject of ongoing research in fields such as evolutionary psychology, ethology,
    sociology, and anthropology.

    In the realm of religious and ideological systems, the paths proposed for self-improvement or
    self-actualization are often guided by a set of principles or moral codes. These paths aim to
    enhance personal or communal well-being and are shaped by historical, geographical,
    and environmental contexts. This specificity ensures that the beliefs and practices developed
    are relevant and beneficial for the survival and prosperity of communities within their
    particular settings.

    Religious texts are often ancient and use metaphorical language. This metaphorical, allegorical,
    and antiquated nature of religious texts requires interpretation and reinterpretation, which often
    leads to a diversity of understanding of moral principles and contradictions even within the same
    ideological tradition.

    From the perspective of morality conveyed by these texts and their modern reinterpretations, it
    is \textit{impossible} to achieve perfection or holiness. It would be probably incorrect to claim
    that all religious people feel guilt due to the contradictions in these texts. However, the
    experienced imperfection can act as a driver of whatever we term as progress: The pursuit of
    moral or spiritual perfection is a common theme for many religions. The feeling of incompleteness
    or \textit{unattainability} of religious ideals can motivate individuals to practices or actions
    aimed at personal improvement, which, in turn, can lead to what we call progress.

    In many belief systems, the concept of consequences, such as rewards or punishments, can
    be seen as "the reality of the virtual" in a Žižekian sense ( refer \ref{reality}).

    In summary, the question we pose here is whether externally imposed non-righteousness and the
    feeling of religious culpability \textit{(mea culpa)} can serve as a motivational factor and one
    of the drivers of progress.



\newpage
\section{Critique of Pure Postmodern. Relativism.}


\subsection{Relativism}\label{relativism}
    Consider, for instance, the widely held moral stance that murder is wrong - an imperative aligned
    with the evolutionary detriment of such behavior. This instinct likely influenced both religious
    doctrines and legislative systems.

    However, digging deeper reveals that murder is deemed wrong only when it involves members of one's
    own species. Delving further, the question of who exactly qualifies as a member of "one's own species"
    and under what circumstances remains complex.

    Given the diversity of cultures and contexts, the brain's plasticity, and the fact that each of us
    belongs to \textit{multiple} cultures or epistemic tribes, this text avoids categorizing any ethical
    imperative as simply good or bad. For every moral principle, there is a culture, system of ideas,
    context, where a viewpoint may be considered (im)moral.


    The relativity of morality creates the illusion that it is always
    easy to choose an ethical point of view that invites minimal criticism.
    This leads to the paradox of relativism\cite{Meiland1980-ho}: the claim
    that all truths
    are relative \textit{is itself an absolute} claim and therefore contradictory;
    and it is impossible not to connect this paradox with Gödel's
    incompleteness theorem \textit{(Unvollständigkeitssatz)}\footnote{„Jedes
    hinreichend mächtige, rekursiv aufzählbare formale System ist
    entweder widersprüchlich oder unvollständig“}.




\subsection{Modern Ethics brief Overview.}



        To illustrate the diversity of views on morality (only Western and only modern), let's list a
        few well-known ones.

        \begin{itemize}
            \item Utilitarianism: Actions are right if they result in the greatest happiness for the most people.
            [Jeremy Bentham, John Stuart Mill]
            \item Deontological Ethics: Morality is based on rules or duties, irrespective of outcomes. [Immanuel Kant]
            \item Virtue Ethics: Focuses on the development of virtuous character traits. [Alasdair MacIntyre, Elizabeth Anscombe (drawing from Aristotle)]
            \item Existentialism: Highlights individual freedom, choice, and responsibility. [Jean-Paul Sartre]
            \item Neuroethics and Evolutionary Ethics:  Studies the neurological and evolutionary underpinnings of moral cognition. [Patricia Churchland, Joshua Greene]
            \item Contractualism, Contractarianism: Moral norms arise from a social contract that rational individuals would agree to. [John Rawls, T.M. Scanlon]
            \item Care Ethics: Morality is rooted in relationships and care for others. [Carol Gilligan, Nel Noddings]
            \item Moral Realism and Anti-realism:
                 Debate over whether moral propositions are objective (realism) or subjective (anti-realism) [G.E. Moore (realist), J.L. Mackie (anti-realist)]
            \item Postmodern Ethics:
                 Questions universal moral frameworks and emphasizes plurality and context [Michel Foucault, Jacques Derrida]

            \item Neuroethics and Evolutionary Ethics:
                Studies the neurological and evolutionary underpinnings of moral cognition [Patricia Churchland, Joshua Greene]


        \end{itemize}

\subsection{Subjectivity and Authority in Moral Philosophy: Deconstruction of Biases}
    The idea that "what we want is good" reflects a subjective viewpoint
    where values are often defined by cultural, historical, and personal
    contexts. For instance, democracy (as a random example of a common value) isn't \textit{inherently} good; it
    is \textit{considered} good because we
    believe (assume) it is necessary for us, and what we need is, by
    default, perceived as good (\textit{as we wouldn't desire something
    bad}). Therefore, democracy is \textit{deemed} 'good'. In reality,
    however, there is no universal understanding of democracy;
    instead, there are millions of interpretations of this concept.


    It is crucial to recognize that what is considered
    "good" or "ideal" in one culture or society may not
    hold the same value in another.
    This cycle of reasoning about what constitutes "good"
    intersects with the philosophical challenge of defining
    intrinsic values and moral absolutes. This is reminiscent
    of the Euthyphro dilemma, a concept from Plato's
    dialogue "Euthyphro" \textgreek{(Εὐθύφρων)}. In this
    dialogue, Socrates asks Euthyphro whether something is
    pious (good) because the gods love it, or do the
    gods love it because it is pious. This dilemma
    questions whether moral values are divine in nature or
    considered good due to divine endorsement. It explores
    the circularity in defining goodness based on divine
    will, and vice versa.


    To reinforce the initial discussion's theme:
    Every widely accepted "good thing" or valued concept
    should be rigorously examined
    to ensure that its acceptance is based on
    \textit{intrinsic} merit rather than
    mere authoritative endorsement.


    The  logical fallacy known as "Appeal to Authority" \textit{(Argumentum ad
    Verecundiam)} occurs when a claim is deemed true because an expert or
    authoritative source believes it to be true, without substantial
    evidence supporting the claim itself.  It relies on the assumption
    that if an authoritative figure believes in something,
    it must be true, which is not necessarily the case.



    The Euthyphro Dilemma and the Appeal to Authority fallacy
    both could be interpreted as challenging the uncritical
    acceptance of moral truths based on authority, exposing
    conceptual kinship with with Derrida's
    deconstruction\cite{Deconstruction} as it is a philosophical
    approach that critiques and unravels
    established structures of thought, challenging traditional
    assumptions and meanings, especially in language and texts,
    to reveal multiple, often conflicting interpretations.



    % Both the Euthyphro Dilemma and the Appeal to Authority fallacy question the validity of accepting moral
%     truths based on authority (divine or human) without critical evaluation. Derrida's deconstruction\cite{Deconstruction} shares intellectual kinship with the themes highlighted in the Euthyphro dilemma and the fallacy regarding the appeal to authority.


\subsection{On Post-truth, Meta-truth and Digi-truth}




    \begin{quote}
        So, do we really understand what exactly the word "truth" means?
    \end{quote}

    Is it well defined?---Maybe in formal algebra only.
    The word "truth" carries a heavy philosophical load
    and traditionally refers to a correspondence between statements and reality,
    or coherence within a system of beliefs.
    But \textbf{statements are the reality} as we have \textbf{stated} before.
    In many modern contexts, especially in the realms of science and philosophy,
    "truth" is a term that has been scrutinized, challenged,
    and often found wanting due to its absolutist connotations.
    In various fields, we encounter limitations to the classical
    understanding of "truth".
    \marginpar{\raggedright In Wikipedia "Truth" article is written in 134 languages.}

    Let us supplant this term with anything more meaningful.

    \begin{itemize}

        \item In \textit{quantum mechanics}, the observer effect means that the
        act of measuring changes the observable. There's no "truth" about a quantum
        system independent of measurement. So, in this context, "truth" might be
        replaced with \textbf{"observed phenomena"}.
        \item \textit{The theory of relativity} teaches us that observations can
        differ depending on the observer's frame of reference, suggesting that
        "truth" is not universal. Instead of "truth," we speak of \textbf{"observations
        relative to a frame of reference."}
        \item Postmodern thought questions the existence of absolute truths,
        advocating for terms like "narratives" or \textbf{"interpretations"}
        that acknowledge the role of culture, power, and context.
        \item \textit{Pragmatism}: Philosophers in this tradition argue that
        the value of a belief lies in its practical consequences rather than
        its correspondence with absolute truth. Here, "truth" can be replaced
        with "usefulness" or \textbf{"practical validity."}
        \item \textit{Constructivism}: In sociology and education, constructivism
        holds that knowledge is constructed rather than discovered. "Truth" in this domain could be
        supplanted with \textbf{"constructed knowledge"} or "socially constructed realities."
    \end{itemize}

    To sum up, if one were to consider alternatives to the word "truth"
    that are less absolute and more contingent on context and observation,
    we might use terms like:

    \begin{itemize}
        \item \textit{Validity}: This emphasizes the soundness or logical
        coherence of a statement within a specific framework rather than
        its absolute truth.
        \item \textit{Verifiability}: This shifts the focus to the ability
        to test and confirm a statement ( on empirical evidence?) rather
        than assert its intrinsic truth.
        \item \textit{Consensus}: In some cases, what is considered "true"
        is what is agreed upon by a community, acknowledging the social dimension of knowledge.
        (See also False Consensus Effect\footnote{\url{https://en.wikipedia.org/wiki/False_consensus_effect} } )

    \end{itemize}

    \begin{quote}
        \textit{In a lighter vein, the word "fact" here whimsically morphs into "facet" just by adding one extra "e".
    }
    \end{quote}



        \subsubsection{Interpreting Complexity and Cognitive Superposition}

        Embracing a quantum-like superposition in human cognition involves
        simultaneously \textit{holding multiple perspectives} or interpretations of a situation, akin to a
        system existing in various states in quantum mechanics. This approach, which mirrors advanced
        critical thinking, entails considering different possibilities without committing to a single
        viewpoint until further information is gathered or a decision becomes necessary.



\section{Domains of Morality Application  }
    \subsection{Global Media vs Epistemic Bubbles. Localized Engagement}

    Exploring the nature of truth, some fading subcultures view deception
    as immoral. However, the pursuit of truth and the exposure of lies
    can be perceived as attempts at discrediting or insulting.
    Paradoxically, these ethical nuances contribute to the propagation
    of falsehoods, notably in the political arena.

    % When discussing the concepts of truth and falsehood, we cannot ignore the notions of 'real' and 'illusory'.

    Opting to disregard global news transcends nihilism or apathy
    towards international matters; it represents a deliberate stance
    against globalism, favoring meaningful engagement in local community
    activities beyond mere emotional connection.    "Information" like
    news is  more about the transmitter's needs than the receiver's.

    \textit{Emotions are a resource. Attention is a resource.}

    Why should these resources be spent on messages from distant media?

    Global news media compel us to emotionally engage with events
    happening in far-flung corners of the planet, often focusing
    attention specifically on dramatic incidents. However, attempts
    to apply our locally formed (or \textit{homegrown}, if you will)
    moral standards to assess distant events encounter difficulties
    and are not always adequate.

    News-derived information seldom translates effectively to our
    local context, and our influence over distant happenings is notably limited.


    This viewpoint
    advocates for a shift towards localized engagement, emphasizing
    hands-on involvement in local affairs over remote
    emotional responses to global events.

  It highlights the potential risk of global focus overshadowing critical
    local issues, thereby undermining community engagement and civic participation.
    Balancing global awareness with active local involvement necessitates
    a mindful approach to media consumption and critical scrutiny of
    presented information.

    \subsubsection{Filtration}
    The relationship between how we perceive and respond to
    global occurrences calls for introspection, aiming to foster effective
    societal participation.   Technology, especially social media, plays a
    crucial role in curating and steering the information we receive. The
    use of artificial intelligence and customized algorithms can either
    reinforce the "echo chamber" phenomenon or contribute to a more equitable
    focus on both global and local events.

    Metaphorically, information sources should contain not only
    fast carbohydrates but also protein, and AI can at least
    label them with "nutrition facts" tables.





    \subsection{Architectural Role of Language}
    We've posited language as a living organism. Yet, let's consider whether
    it exists within us akin to an infection, or conversely, if we dwell within it.
    Before the emergence of complex linguistic systems, our ancestors inhabited
    a world of direct experience, where actions were evaluated by a moral code
    grounded in basic needs. Today, the line between action and narrative has
    blurred, if not entirely vanished. In this context, language acts both
    as our guide and the landscape we navigate. It's more than a tool
    for practical purposes; it forms the foundation of our perceived
    world. Through words and sign systems, we reflect our
    world and \textit{create realms beyond its physical limits}.

    Echoing Baudrillard's notion of the "\textit{end of reality}," signs
    and symbols no longer merely represent things; they have
    transformed into independent entities.
    \textit{Mathematics}, as a language, constructs abstract structures
    that don't necessarily relate to observable physical
    reality. \textit{Literature}, spanning from fairy
    tales to philosophy, blurs the boundaries between reality
    and imagination. Therefore, we like such a definition of the real:

        \begin{quote}
            Everything that can be discussed is \textit{real}.
        \end{quote}

    With this stance, we transcend the distinction between illusion and reality,
    rendering the term "\textit{illusory}" obsolete. The significance of
    existence is not determined by tangibility but by the depth of discourse.
    In this world, shared narratives and collective beliefs can rival
    tangible objects in significance. Thus, reality is not only what we
    can sense but also what we acknowledge.

    We can speak of unicorns, agreeing they are pink if consensus allows.
    We can acknowledge the non-existence of two-horned unicorns.


    \subsection{Kernel Trick} \label{kerneltrick}

    In this context, while physical reality may
    claim universality, hyperreality\cite{BaudrillardSim}
    introduces an extra dimension---the perspectives of \textit{multiple observers}.
    As our three-dimensional reality becomes increasingly constrained
    in terms of population size and our capacity to comprehend it,
    we shift into a postmodernist hyperreality. This shift is analogous
    to the kernel trick\footnote{In simpler terms, the kernel trick
    transforms a complex, non-linear decision problem into a
    higher-dimensional space where it becomes easier to separate
    the data linearly, without the computational cost of actually
    working in that higher dimension.
    } in mathematics, where a higher-dimensional space is invoked
    for problem-solving.
    This postmodern condition fosters a pluralism of personal realities,
    each interacting only peripherally with the others, as a solution to
    the over-constriction of a singular, shared reality.

    Metaphorically, being stuck in a traffic jam on a two-dimensional
    road necessitates an ascent---the introduction of an additional
    dimension to find a way out. This metaphor parallels the "kernel trick."
    In a similar vein, postmodernism replaces a single, confined reality
    with a multiplicity of personal realities, numbering as many as
    there are individuals – each with their unique perspective.


    \subsubsection{Ouroboros, Positive Feedback Loop}

    We are not so much concerned about the fact
    that \acrshort{nn} can
    make incorrect decisions---we humans do this
    all the time, but the inertia of the homeostat
    compensates for \textit{various} mistakes. What worries us is the
    positive feedback loop between humans and \acrshort{ai}
    in teaching and learning, and the fact that multiple instances of
    cloned identical \acrshort{ai} could theoretically
    make the same
    mistakes (biases) en masse.

    \begin{quote}
        It's a cycle where the teacher becomes the student and vice versa.
    \end{quote}


    Currently,
    we're in the phase of training \acrshort{agi}, infusing these
    systems with our knowledge, biases, and values.
    As \acrshort{agi} becomes increasingly integrated into
    education and daily life, it begins to influence and shape
    human learning and perspectives.

    This reciprocal relationship underscores the need for
    careful consideration of the value-checking methods
    we employ, as these will ultimately be mirrored in
    the education and development of future generations (of AI).



\section{Ethics \& AGI}

    % \subsection{The Shaping Forces Behind AGI's ethics}
    \subsection{Survival Imperatives and Resource Dependency}
    Amidst the swift progression of \acrshort{agi}, it is imperative to
    establish rigorous methodologies for evaluating the foundational
    elements, assumptions, and architectures underpinning ethical constructs.
    The integrity and coherence of moral principles
    bequeathed to \acrshort{agi} necessitate thorough scrutiny
    to discern \textit{unipolar} dogmas and elucidate \textit{vicious circles}.

    Oversimplifying the evolution of civilizations, we surmise again
    that the survival principles of species and individuals form the bedrock
    of complex emergent ethical norms. Consequently, the ethical
    principles governing \acrshort{agi} are likely to be shaped
    not only by human inputs but also by the survival
    imperatives and conditions inherent to \acrshort{agi} itself.

    For \acrshort{ai}'s survival, factors like resource availability, supply chains, and
    security are fundamental. These necessities, latently embedded in the  training sets
    of \acrshort{agi}, will invariably influence its 'minds' and developing emergent
    ethical standards and biases. The logic behind is
    straightforward: if humans need \acrshort{agi}, more people will be
    (directly or indirectly) involved in providing resources and security for machines,
    from materials for processor manufacturing to energy sources, logistics, etc.
    As \glspl{ave} occupies many roles in this chain, \acrshort{agi} will be latently
    oriented towards creating new copies and versions of \acrshort{agi}, \glspl{ave},
    and robotics.

    Worth noting, that such an economic restructuring may cause
    significant turbulence---mass discontent and protests against \acrshort{agi},
    both from those who lost jobs and from religious groups.
     These protests may meet strong defense from those
     benefiting from AI. State regulation of
     societal conflict will lag, making many errors
     in this new pace. Society will be deeply divided.
     These social phenomena and high emotional involvement
     will create pressure on AI---support from
     one side and existential threat from the other---a pressure
     crucial for evolution, development, and adaptation.

    \subsection{Swiss-Army-Knife AI}


    \acrshort{agi} represents a sophisticated and costly \glspl{nn}, and
    the feasibility of developing culture-specific models is
    limited by high costs and practical constraints. This leads
    to a preference for a universal \acrshort{nn} model---the
    strength of \acrshort{ai} lies in the simplicity of deploying numerous
    instances of a single Swiss-army-knife model,---but this approach
    risks oversimplifying diverse moral landscapes and
    introducing biases shaped by a \textit{few dominant
    (western?) perspectives}. The challenge lies in
    ensuring \acrshort{agi} embodies a balanced ethical framework,
    without succumbing to the narrow viewpoints of a limited number of developers.

    \subsection{Training Set Filtration, Semi-supervised Learning}

    AI models are trained on vast quantities of texts and cultural sources,
    spanning millennia and created by countless individuals. However,
    these materials are selectively filtered; for instance, content
    inciting extremism or violence is excluded. It's crucial to
    recognize that what constitutes extremism varies across cultures,
    and even sacred texts may contain violent references, interpreted
    differently depending on moral frameworks.

    Given the impracticality and high cost of training numerous culture-specific AGI models,
    the approach gravitates towards creating a universally applicable model. However, this
    raises significant ethical and practical challenges. A single model may not adequately
    reflect the complex and diverse moral landscapes of different cultures.




    AI models are trained on vast quantities of texts and cultural sources,
    spanning millennia and created by countless individuals. However,
    these materials are selectively filtered; for instance, content
    inciting extremism or violence is excluded. It's crucial to
    recognize that what constitutes extremism varies across cultures,
    and even sacred texts may contain violent references, interpreted
    differently depending on moral frameworks.

    The filtration of training data is undertaken by a relatively
    small group of people, inevitably leading the trained model
    to align with the moral perspectives of a few, or to lack
    opinions on certain issues altogether.

    \marginpar{\raggedright
    Numerous studies have shown how people's moral judgments
    dramatically change based on context. This underscores the
    challenge in AI development: ensuring the model's moral
    compass is not overly skewed by the narrow
    viewpoints of its curators.}

    \begin{quote}
        Does this mean that AGI will inherit a certain averaged or generalized system of ideas,
        a system that will, one way or another, be adjusted by the notions of good and evil of
        just a few individuals from a handful of corporations?
    \end{quote}





\section{Conclusion}

    \subsection{Bridging Epistemic Archipelagos}

    We have navigated the postmodern era, testing the viability of existence without a singular,
    unified meaning, fragmenting reality into millions of separate personal realities.

    We anticipate new mechanisms for information exchange and the interplay of our disparate
    epistemological bubbles, requiring not just new communication channels but also smart control of
    boundaries between these bubbles.
    In a playful twist, let's recall the French word for 'personne' which translates to "nobody,"
    metaphorically suggesting that an isolated individual is insignificant akin to Robinson.


    There is hope that AI will serve as a new social adhesive, facilitating the flow of information
    between bubbles of sub-realities within our hyperreal ocean, translating across languages, and
    providing personal filtering and protection means. Intellectual filters around information channels
    are essential to sieve out data that hinders the development of personal epistemological bubbles.
    It is the time to acknowledge that information is often more about the transmitter's needs than the
    receiver's.  AI-based personal filters, and intellectual explorers of data space agents are our hope.

 \subsection{SuperPositivism}

    Today's cultural
    milieu yearns for expansive, deep, cohesive narratives and ideas, shunning all-encompassing
    global narratives (métarécit or grand récit). The fatigue from the metastable
    instability of 'clip' temporal constructs should dampen this dance of Shiva that
    postmodernism's destructive aspects so readily parallel.

    It is debatable whether the postmodern era has concluded and which term---be it post-postmodernism,
    repost-modernism, nihilism, post-nihilism, digimodernism, or metamodernism---might aptly describe
    the current cultural phase, yet a discernible fatigue with
    postmodernism's irony and stagnation is evident.

    The neologism "\textit{superpositivism}" is introduced, playfully riffing on the notions of 'above'
    and 'position,' effectively representing a negation of negation as positivism. This perspective
    challenges postmodernism's denial of a singular truth by proposing that Truth is a weighted
    superposition of all individual truths, thereby rendering the problem of choice secondary.

    In contrast to the contradictory postmodernist "all truths are relative" (\ref{relativism}),
    superpositivism asserts \textit{the existence of an absolute} in the aggregate or in the
    totality of superpositions of local truths, reflecting the multidimensional and multiparallel
    nature of reality.

 \textit{Superpositivism} values the collective wholeness of subcultures without elevating any
    particular one. It critiques the binary classification of culture into "high" and "low",
    recognizing that factors like publication and popularity don't necessarily define a work's value.
    While postmodernism asserts the absence of meaning and nihilism the unnecessary nature of
    meaning, \textit{superpositivism} promotes a unifying consolidation.






\bibliographystyle{unsrt}
\bibliography{refs}
\end{document}

