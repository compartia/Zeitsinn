% https://www.overleaf.com/project/657c60bf2ef1fbba865e8e75
\documentclass[11pt,a4]{article}

\usepackage{amsmath}
\usepackage{amssymb}
\usepackage{amsfonts}
\usepackage{array}
\usepackage{mathrsfs}
\usepackage{multirow}
\usepackage{siunitx}
\usepackage{acro}
\usepackage[acronym]{glossaries}

% \usepackage{acronyms}
\usepackage{booktabs}
\usepackage[LGR, T1]{fontenc}
\usepackage[utf8]{inputenc}   % utf8 is required

% \usepackage{fancyhdr}
% % \pagestyle{fancy}
% \lhead{M Python}

% \usepackage{marginnote}

% \usepackage{geometry}
%  \geometry{
%  a4paper,
%  % marginpar=220pt
%  }


% \usepackage[english, russian]{babel}

% \setlength\topmargin{-1.1in} \addtolength\textheight{2.1in}
% \addtolength{\oddsidemargin}{-0.2in}
% \addtolength{\evensidemargin}{-0.1in} \textwidth 5.8in
\newcounter{questioncounter}
\newcounter{equestioncounter}
% \setlength\parskip{10pt} \setlength\parindent{0in}
% \newcommand{\bea}{\begin{eqnarray*}}
% \newcommand{\eea}{\end{eqnarray*}}
% \newcommand{\beao}{\begin{eqnarray}}
% \newcommand{\eeao}{\end{eqnarray}}

\newcommand{\textgreek}[1]{\begingroup\fontencoding{LGR}\selectfont#1\endgroup}

% \newcommand{\no}{\noindent}


\newacronym{ai}{AI}{artificial intelligence}
\newacronym{agi}{AGI}{artificial general intelligence}
\newacronym{nn}{NN}{neural network}
\newacronym{ave}{AVE}{artificial virtual employee}
\newacronym{llm}{LLM}{large language model}

\makeglossaries

\begin{document}
\title{The Confluence of Ethics, Language and AI}

\author{Artem K. Zaborskiy}


\maketitle

% https://www.overleaf.com/project/657c60bf2ef1fbba865e8e75



\begin{abstract}



This essay is an exploration of the interplay between language, ethics,
and societies. It delves into how language reflects and shapes our
observer-dependent realities, the role of \glspl{llm} in ethics and
linguistics development, the ethics of speech acts, and the influence
of language in the political sphere. The essay aims to unravel the
complex relationships between inherited morality and learned linguistic
structures, prompting a reevaluation of traditional notions of
morality and truth in the light of \gls{ai} development,
exploring the pluralistic nature of ethics without trying
to define what is moral and what is not.
\end{abstract}

\tableofcontents
% \printacronyms
\printglossary[type=\acronymtype]




% --------------- ---------------

\section{Introduction}

    Albert Schweitzer described life as a precious and mysterious
    gift\cite{Schweitzer}. His concept of \textit{“Ehrfurcht vor dem Leben,”}
    or “Reverence for Life,” advocates a profound respect for all living beings.
    Yes, every creature is endowed with a will to live, a product of
    natural selection, but evolutionary mechanisms are indifferent
    – every species eventually becomes extinct, making way for new ones.
    \par
    The lifespan of an individual, or even a whole species, is
    laughably short compared to the known history of the universe.
    It seems logical, then, to imbue this brief existence with
    meaning and purpose within the vast, almost infinite
    evolution of the universe.
    \par
    Perhaps it is this contradiction, this very tension
    between the poles of guaranteed mortality and a
    strong will to live, that grounds the pursuit of
    purpose, meaning, and ways of living. This search
    is solidified in ethics and morality.

    \par
    If life is merely a tiny link in the long chain of
    existence, what exactly do we take from the past
    and carry into the future? \textit{What criteria
    do we apply to discern what is worth leaving in the past?}

    \subsection{Modern Ethics Overview}

        Brief overview of modern western philosophical views on morality:

        \begin{itemize}
            \item Utilitarianism: Actions are right if they result in the
            greatest happiness for the most people. [Jeremy Bentham, John Stuart Mill]

            \item Deontological Ethics: Morality is based on rules or duties, irrespective of outcomes. [Immanuel Kant]
            \item Virtue Ethics: Focuses on the development of virtuous character traits. [Alasdair MacIntyre, Elizabeth Anscombe (drawing from Aristotle)]
            \item Existentialism: Highlights individual freedom, choice, and responsibility. [Jean-Paul Sartre]
            \item Neuroethics and Evolutionary Ethics:  Studies the neurological and evolutionary underpinnings of moral cognition. [Patricia Churchland, Joshua Greene]
            \item Contractualism, Contractarianism: Moral norms arise from a social contract that rational individuals
                  would agree to. [John Rawls, T.M. Scanlon]


            \item Care Ethics: Morality is rooted in relationships and care for others.
                [Carol Gilligan, Nel Noddings]

            \item Moral Realism and Anti-realism:
                 Debate over whether moral propositions are objective (realism)
                or subjective (anti-realism)
                [G.E. Moore (realist), J.L. Mackie (anti-realist)]

            \item Postmodern Ethics:
                 Questions universal moral frameworks and emphasizes plurality and context
                [Michel Foucault, Jacques Derrida]

            \item Neuroethics and Evolutionary Ethics:
                Studies the neurological and evolutionary underpinnings of moral cognition
                [Patricia Churchland, Joshua Greene]


        \end{itemize}


    TODO:


\section{Cultural and Genetic Dynamics, Foundations of Ethics}

    Let's approach dogmas and well-known moral principles not as
    immutable, eternal axioms bestowed from above, but as constructs
    with discernible causes, aiming to uncover their origins rather than
    justifying them.
    \par
    Investigating the foundations of any moral principle or law often
    reveals that such principles are evolutionarily and survivalistically
    advantageous. This suggests a correlation between ethical
    norms and their evolutionary benefits. It seems plausible
    that behaviors beneficial for
    survival initially become ingrained in our genes, manifesting as
    instincts, and are subsequently refined through cultural superstructures.
    These behaviors are then articulated in language,
    codified in legislation, echoed in sacred texts, legends,
    instructive fairy tales, and even moralistic narratives in
    Disney animations, etc.

    \par
    In essence, human language and DNA share a common goal: the \textit{transmission} of information across \textit{space and time}\footnote{\\- A typical Bible text file, without special formatting, takes about 4-5 MB. \\
    - Human DNA is often quantified in terms of base pairs.
    The human genome contains about 3 billion base pairs.
    The storage capacity of DNA is usually estimated based on the
    fact that each base pair can contain
    2 bits of information (since there are four possible nucleotides: A, T, C, G).
    Thus, the total amount of information that can be stored in the human
    genome is about 750 MB, which corresponds to approximately 150–200 copies of
    the Bible.\\
    - A speculative and unscientific estimate of the capacity
    of the human brain is tens of terabytes, which is
    equivalent to several million books.}.

    \par
    Both mechanisms are fundamental to inheritance and
    the sustenance of life, making \textit{this parallel noteworthy.}





    \subsection{Development of Means of Knowledge Transfer on a Timeline}

 \par
    The most critical aspect to emphasize is the \textit{relative speed} of
    development between the two mechanisms. Genetic structure evolves extremely
    slowly compared to the rapid pace of advancements in knowledge transmission
    technologies, particularly language. Consequently, instincts change at a
    sluggish rate, whereas the conditions, environment, and context in
    which our animal instincts still operate are transforming with a
    completely different dynamic.\\
    \par
        To illustrate this dynamic, key historical epochs can be briefly summarized:


        \begin{itemize}
            \item[-] Drawing from archaeological evidence and the anatomical analysis of the human hyoid bone, it is posited that complex linguistic abilities might have been present in early members of the Homo genus, potentially in species such as Homo erectus or Homo heidelbergensis \cite{Capasso2008AHE}.
            \item[-] Homo sapiens, our own species, has existed for an estimated 200-300 thousand years\cite{Vidal}.
            \item[-] On the timeline of human evolution, formal literacy is a relatively recent phenomenon, having emerged around 5,300 years ago with the advent of Sumerian cuneiform\cite{Walker}.
            \item[-] The Internet was adopted around 50 years ago.
            \item[-] LLMs are 1-2 years old
        \end{itemize}


        \textit{200,000 – 5,000 – 50 – 2:}
        greatly simplifying the complex, non-linear nature of human progress and technological progress, one can notice that each subsequent era is on average 50 times shorter than the previous one.



\subsection{The Presumption of Comprehension}
    TODO:

    \subsubsection{Patterns through Generations}
        \par
        Historically, life unfolded through predictable patterns. A son's path often echoed his father's; communities upheld uniformity. Within this regularity, the stories they lived by were nearly similar. Comprehension thrived on a shared matrix of semantic reference points, fostering a consistent flow of knowledge.

    \subsubsection{Enhancing the Ambiguity}
        \par
        In today's world, brimming with diverse narratives, linguistic variety, and a stream of digital imagery, including AI-generated media, the purity of terms is called into question. One person's understanding, shaped by cinema, literature, and personal experience, forms a collage distinct from another person's collage, assembled under completely different influences.
        \par
        As language expands into new domains, its precision fades. With each new context, a word's meaning multiplies, becomes diluted, enhancing its \textit{ambiguity}.




        Despite the widening gaps in understanding, there persists an unspoken belief in our ability to understand each other.

        Consider the statement, "I believe in God." Delving into its meaning reveals that the speaker's conception of "God" and "belief" is a unique patchwork quilt, pieced together from various threads of past life and experiences (which off course echoes Derrida's view that meaning is not absolute but fluid and open to various interpretations \cite{Deconstruction}).

        \par
        It's not merely that we pretend to grasp each other's meanings; we seldom even acknowledge the prospect of \textit{being misunderstood}. \\
        \\
        \textit{\textbf{We do not understand that we are misunderstood}.}  \\

        \par
        This diversity is not a problem that demands resolution, but rather a \textit{characteristic} of language to be \textit{acknowledged}. Conversely, people tend to cooperate more readily when they believe they share more similarities than differences.

    \subsubsection{Interpreting Complexity and Cognitive Superposition}
        \par
        Terms are often transposed from old contexts into new ones, shaped by emerging technologies.
        For instance,  the initial meanings of "kosmos" (\textgreek{κόσμος}) in ancient Greek culture encompassed concepts of order, arrangement, and adornment (cosmetic). Over time, these meanings evolved through philosophical discourse to the modern understanding of "cosmos" as the universe seen as a well-ordered, harmonious whole.
        \par
        The diffusion of  terms can lead to the creation of philosophical pseudo-problems and paradoxes, which might be resolved simply by redefining terms or frame of reference.
        \par

        Take as an illustration the centuries-old debate surrounding the
        omnipotence paradox expressed in the question: \textit{"Can an omnipotent being create a stone so heavy that it cannot lift it?"} This discussion can be re-framed to highlight its terminological ambiguity and even absurdity.
        At what point exactly does the observer measure the omnipotence of the Being - before, during or after the creation of the stone?
        Further pursuing this line of \textit{reductio ad absurdum}, from the stone's frame of reference, the concept of lifting becomes irrelevant, thereby challenging the anthropocentric biases traditionally embedded in the paradox’s interpretation.
        The enduring value of such paradoxes is in the depth and breadth of discussion it generates. \textit{In fact we don't really need any single solution to it.}
        \par

        Embracing a quantum-like superposition in human cognition involves simultaneously \textit{holding multiple perspectives} or interpretations of a situation, akin to a system existing in various states in quantum mechanics. This approach, which mirrors advanced critical thinking, entails considering different possibilities without committing to a single viewpoint until further information is gathered or a decision becomes necessary.

\subsection{TODO: transition}
TODO:

\subsection{Language as an Organism}
    The dynamics of survival among various systems of ideas and languages can be viewed as akin to the evolution of biological species. Metaphorically speaking, different nations exist in symbiosis with their culture, competing with other nations and their ideologies for resources.
    The primary resource has always been land. Culture, language, mentality, and moral norms are largely determined by the characteristics of the landscape and the type of nature.

    \par
    The development of language mirrors the evolution of a living system, or organism. For the evolution of any complex system, maintaining homeostasis requires two essential components or tendencies: a certain degree of staticness (a) and inertia, alongside dynamics (b), as discussed by Norbert Wiener in his book "Cybernetics: Or Control and Communication in the Animal and the Machine" [source]. The evolution of human language exemplifies the interplay between these components - staticness and dynamics. Natural linguistic changes are moderated by the process of standardization.

    \subsection{The Battle of Dialects}
    As the population increased, the complexity of managing public order also grew, necessitating clear and unambiguous legislation. This need catalyzed a shift towards linguistic unification. The precise formulation of legal terminology became crucial in ensuring a uniform interpretation and application of laws in various regions.
    \par
    As a result, the dialect used for official purposes, particularly in legal documents and procedures, often gained significant prestige. It evolved into the standard — a 'prestigious' dialect\footnote{It is crucial to recognize that a language's high style or a dialect distinct from "rural" or "provincial" ones, often featured in media, is not inherently superior to other regional dialects. Its prominence is largely a result of historical and sociopolitical factors rather than linguistic merit. The dialect in which laws are written, perceived as 'standard' or prestigious, gains its status more from its association with power and authority than from its intrinsic linguistic qualities.}, establishing a template for official communication and embedding itself as the norm.

    \subsection{Ambiguity versus Specificity}
    In language development, we observe a tension between the increase in ambiguity and the rise of specificity - some aspects crystallize while others dissolve. Discussing language in the context of morality is vital to ensure that the doctrines represented in narratives are correctly understood in the modern context, a concern particularly pertinent in the realm of \acrshort{ai} development.
    \par
    Recent advancements in \glspl{llm} can act as linguistic anchors.
    Utilizing computational speed, \glspl{llm}  can track language changes or even resolve discrepancies in definitions, offering an "updated dictionary" that reflects unified interpretations of terms. While \glspl{llm}  can enhance mutual understanding and provide a basis for consensus, it's important to recognize their potential for homogenizing the conceptual framework. This linguistic globalization, from an evolutionary perspective, may not be entirely beneficial, but to judge evolution would be overly anthropocentric.

    \section{TODO: transition}

    \subsection{TODO: Deconstruction of Biases}
    The idea that "what we want is good" reflects a subjective viewpoint where values are often defined by cultural, historical, and personal contexts. For instance, democracy isn't inherently good; it is considered good because we believe (assume) it is necessary for us, and what we need is, by default, perceived as good (\textit{as we wouldn't desire something bad}). Therefore, democracy is deemed 'good'. In reality, however, there is no universal understanding of democracy; instead, there are millions of interpretations of this concept.
    \par

    It is crucial to recognize that what is considered 'good' or 'ideal' in one culture or society may not hold the same value in another. This cycle of reasoning about what constitutes 'good' intersects with the philosophical challenge of defining intrinsic values and moral absolutes. This is reminiscent of the Euthyphro dilemma, a concept from Plato's dialogue "Euthyphro" \textgreek{(Εὐθύφρων)}. In this dialogue, Socrates asks Euthyphro whether something is pious (good) because the gods love it, or do the gods love it because it is pious. This dilemma questions whether moral values are divine in nature or considered good due to divine endorsement. It explores the circularity in defining goodness based on divine will, and vice versa.

    \par


    The  logical fallacy known as 'Appeal to Authority' \textit{(Argumentum ad Verecundiam)} occurs when a claim is deemed true because an expert or authoritative source believes it to be true, without substantial evidence supporting the claim itself.  It relies on the assumption that if an authoritative figure believes in something, it must be true, which is not necessarily the case.

    \par

    Both the Euthyphro Dilemma and the Appeal to Authority fallacy question the validity of accepting moral truths based on authority (divine or human) without critical evaluation. Derrida's deconstruction\cite{Deconstruction} shares intellectual kinship with the themes highlighted in the Euthyphro dilemma and the fallacy regarding the appeal to authority.


\subsection{Architectural Role of Language}
        We've posited language as a living organism. Yet, let's consider whether it exists within us akin to an infection, or conversely, if we dwell within it.
        \par
        Before the emergence of complex linguistic systems, our ancestors inhabited a world of direct experience, where actions were evaluated by a moral code grounded in basic needs. Today, the line between action and narrative has blurred, if not entirely vanished. In this context, language acts both as our guide and the landscape we navigate. It's more than a tool for practical purposes; it forms the foundation of our perceived world. Through words and sign systems, we reflect our world and \textit{create realms beyond its physical limits}.

        \par
        Echoing Baudrillard's notion of the '\textit{end of reality},' signs and symbols no longer merely represent things; they have transformed into independent entities.
        \textit{Mathematics}, as a language, constructs abstract structures that don't necessarily relate to observable physical reality. \textit{Literature}, spanning from fairy tales to philosophy, blurs the boundaries between reality and imagination.

        \par
        In our definition, anything\textit{ discuss-able} is \textit{real}. With this stance, we transcend the distinction between illusion and reality, rendering the term '\textit{illusory}' obsolete. The significance of existence is not determined by tangibility but by the depth of discourse. In this world, shared narratives and collective beliefs can rival tangible objects in significance. Thus, reality is not only what we can touch but also what we acknowledge.
        \par
        We can speak of unicorns, agreeing they are pink if consensus allows. We can assert the non-existence of two-horned unicorns.

\subsection{On Truth (postmodern)}
Postmodernism dismantles foundational concepts that have historically supported the global ethical narrative, notably the concept of "truth." Rather than offering alternatives to the deconstructed framework, postmodernism arises as a response to technological progress and the corresponding evolution of language.

In this context, while physical reality may claim universality, hyperreality introduces an extra dimension - the perspectives of multiple observers. As our three-dimensional reality becomes increasingly constrained in terms of population size and our capacity to comprehend it, we shift into a postmodernist hyperreality. This shift is analogous to the 'kernel trick' in mathematics, where a higher-dimensional space is invoked for problem-solving.
This postmodern condition fosters a pluralism of personal realities, each interacting only peripherally with the others, as a solution to the over-constriction of a singular, shared reality.

Metaphorically, being stuck in a traffic jam on a two-dimensional road necessitates an ascent - the introduction of an additional dimension to find a way out. This metaphor parallels the 'kernel trick.' In a similar vein, postmodernism replaces a single, confined reality with a multiplicity of personal realities, numbering as many as there are individuals – each with their unique perspective.


\par
\textit{So, do we really understand what exactly the word “truth” means?} \\
\par
Is it well defined? -- Maybe in formal algebra only. The word "truth" carries a heavy philosophical load and traditionally refers to a correspondence between statements and reality, or coherence within a system of beliefs. But \textbf{statements are the reality} as we have \textbf{stated} before.
In many modern contexts, especially in the realms of science and philosophy, "truth" is a term that has been scrutinized, challenged, and often found wanting due to its absolutist connotations.
In various fields, we encounter limitations to the classical understanding of "truth".
\marginpar{\raggedright In Wikipedia "Truth" article is written in 134 languages.}


Let us supplant this term with anything more meaningful.

\begin{itemize}

    \item In \textit{quantum mechanics}, the observer effect means that the act of measuring changes the observable. There's no "truth" about a quantum system independent of measurement. So, in this context, "truth" might be replaced with \textbf{"observed phenomena"}.
    \item \textit{The theory of relativity} teaches us that observations can differ depending on the observer's frame of reference, suggesting that "truth" is not universal. Instead of "truth," we speak of \textbf{"observations relative to a frame of reference."}
    \item Postmodern thought questions the existence of absolute truths, advocating for terms like "narratives" or \textbf{"interpretations"} that acknowledge the role of culture, power, and context.
    \item \textit{Pragmatism}: Philosophers in this tradition argue that the value of a belief lies in its practical consequences rather than its correspondence with absolute truth. Here, "truth" can be replaced with "usefulness" or \textbf{"practical validity."}
    \item \textit{Constructivism}: In sociology and education, constructivism holds that knowledge is constructed rather than discovered. "Truth" in this domain could be supplanted with \textbf{"constructed knowledge"} or "socially constructed realities."
\end{itemize}

To sum up, if one were to consider alternatives to the word "truth" that are less absolute and more contingent on context and observation, we might use terms like:
Validity: This emphasizes the soundness or logical coherence of a statement within a specific framework rather than its absolute truth.

\begin{itemize}

    \item \textit{Verifiability}: This shifts the focus to the ability to test and confirm a statement ( on empirical evidence?) rather than assert its intrinsic truth.
    \item \textit{Consensus}: In some cases, what is considered "true" is what is agreed upon by a community, acknowledging the social dimension of knowledge.

\end{itemize}

\textit{In a lighter vein, the word “fact” here whimsically morphs into “facet” just by adding one extra symbol.
} \\
\par

Exploring the nature of truth, some fading subcultures view deception as immoral. However, the pursuit of truth and the exposure of lies can be perceived as attempts at discrediting or insulting. Paradoxically, these ethical nuances contribute to the propagation of falsehoods, notably in the political arena.\\

When discussing the concepts of truth and falsehood, we cannot ignore the notions of 'real' and 'illusory'.

    \subsection{The Reality of the Virtual}

        In Slavoj Žižek's exploration of Lacanian psychoanalysis, he introduces the concept of "\textit{the reality of the virtual}" which hinges on the interplay between Lacan's three realms of human reality: the Symbolic, the Imaginary, and the Real.
        In this framework, the potency of punishment derives from its indeterminate nature \cite{Borretzen2012-tx}. It is the uncertainty, the not-knowing, that gives rise to the fear of punishment.

        \par
        If a generalized eschatological infernal realm were meticulously described, it might transition from an anxiety-inducing abstraction to a tamed reality, clearly defined and consequently less formidable. It is the \textit{nebulous}, the undefined, that truly haunts us and ought to remain amorphous to preserve its capacity to instill dread. These threats linger best in the space of the virtual, potent and chilling in their ambiguity. In this vein, one might jest that Dante, in his vivid portrayal of the Inferno, may have unwittingly pilfered from us the more terrifying specters of the unknown; he gave us a map where perhaps we were better off with uncertainty.













\subsection{Ethics of \acrshort{agi} }

\subsubsection{I. Swiss-Army-Knife}

In the era of advanced \acrshort{agi}, scrutinizing the foundations of
morality becomes essential, as \acrshort{ai} systems inherit
these principles.
\acrshort{agi} represents a sophisticated and costly \glspl{nn}, and
the feasibility of developing culture-specific models is
limited by high costs and practical constraints. This leads
to a preference for a universal \acrshort{nn} model -- the
strength of \acrshort{ai} lies in the simplicity of deploying numerous instances of a single Swiss-army-knife  model,--  but this approach
risks oversimplifying diverse moral landscapes and
introducing biases shaped by a \textit{few dominant
(western?) perspectives}. The challenge lies in
ensuring \acrshort{agi} embodies a balanced ethical framework,
without succumbing to the narrow viewpoints of a limited number of developers.

\subsubsection{II. Training Set Filtration, Semi-supervised Learning}
AI models are trained on vast quantities of texts and cultural sources,
spanning millennia and created by countless individuals. However,
these materials are selectively filtered; for instance, content
inciting extremism or violence is excluded. It's crucial to
recognize that what constitutes extremism varies across cultures,
and even sacred texts may contain violent references, interpreted
differently depending on moral frameworks.
\par
The filtration of training data is undertaken by a relatively
small group of people, inevitably leading the trained model
to align with the moral perspectives of a few, or to lack
opinions on certain issues altogether.
\par
Numerous studies have shown how people's moral judgments
dramatically change based on context. This underscores the
challenge in AI development: ensuring the model's moral
compass is not overly skewed by the narrow
viewpoints of its curators.

\subsubsection{III. Ouroboros, Positive Feedback Loop}
    \par
    Our concern isn't so much that \acrshort{nn} might make incorrect decisions --
    humans do this regularly – but rather the positive feedback loop
    between people and \acrshort{ai} in teaching and learning. Currently,
    we're in the phase of training \acrshort{agi}, infusing these
    systems with our knowledge, biases, and values.
    As \acrshort{agi} becomes increasingly integrated into
    education and daily life, it begins to influence and shape
    human learning and perspectives.
    \par
    This reciprocal relationship underscores the need for
    careful consideration of the value-checking methods
    we employ, as these will ultimately be mirrored in
    the education and development of future generations.
    \par
    It's a cycle where the teacher becomes the student and vice versa,
    highlighting the intricate interplay between human and artificial
    intelligence in shaping knowledge and values.


\subsubsection{IV. TODO: title}



    Oversimplifying the evolution of civilizations, we surmise again that the survival principles of species and individuals form the bedrock
    of complex emergent ethical norms. Consequently, the ethical
    principles governing \acrshort{agi} are likely to be shaped
    not only by human inputs but also by the survival
    imperatives and conditions inherent to \acrshort{agi} itself.


    \par
    For \acrshort{ai}'s survival, factors like resource availability, supply chains, and security are fundamental. These necessities, latently embedded in the  training sets of \acrshort{agi}, will invariably influence its 'minds' and developing emergent ethical standards and biases.


    The logic is straightforward: if humans need \acrshort{agi}, more people will be (directly or indirectly) involved in providing resources and security for machines, from materials for processor manufacturing to energy sources, logistics, etc. As \glspl{ave} occupies many roles in this chain, \acrshort{agi} will be latently oriented towards creating new copies and versions of \acrshort{agi}, \glspl{ave}, and robotics.

    \par
    It's important to note that such an economic restructuring will cause significant turbulence — mass discontent and protests against AI, both from those who lost jobs and from religious groups. These protests will meet strong defense from those benefiting from AI. State regulation of societal conflict will lag, making many errors in this new pace. Society will be deeply divided. These social phenomena and high emotional involvement will create pressure on AI – support from one side and existential threat from the other, a pressure crucial for evolution, development, and adaptation.


\section{TODO: Conclusion}
    \subsection{TODO: Name}
    We have navigated the postmodern era, testing the viability of existence without a singular, unified meaning, fragmenting reality into millions of separate personal realities.
    \par
    We anticipate new mechanisms for information exchange and the interplay of our disparate epistemological bubbles, requiring not just new communication channels but also smart control of boundaries between these bubbles.
    In a playful twist, let's recall the French word for 'personne' which translates to 'nobody,' metaphorically suggesting that an isolated individual is insignificant akin to Robinson.
    \par
    There is hope that AI will serve as a new social adhesive, facilitating the flow of information between bubbles of sub-realities within our hyperreal ocean, translating across languages, and providing personal filtering and protection means. Intellectual filters around information channels are essential to sieve out data that hinders the development of personal epistemological bubbles. It is the time to acknowledge that information is often more about the transmitter's needs than the receiver's.  AI-based personal filters, and intellectual explorers of data space agents are our hope.


    \subsection{SuperPositivism}
    It is debatable whether the postmodern era has concluded and which term - be it post-postmodernism, repost-modernism, nihilism, post-nihilism, digimodernism, or metamodernism - might aptly describe the current cultural phase. Such classifications remain debatable, yet a discernible fatigue with postmodernism's irony and stagnation is evident.
    \par
    The neologism "\textit{superpositivism}" is introduced, playfully riffing on the notions of 'above' and 'position,' effectively representing a negation of negation as positivism. This perspective challenges postmodernism's denial of a singular truth by proposing that Truth is a weighted superposition of all individual truths, thereby rendering the problem of choice secondary.
    \par
    The postmodernist assertion, or the relativistic paradox, that all truths are relative – itself an absolute statement – is inherently contradictory. In contrast, \textit{superpositivism} posits that the absolute exists in the totality of the superposition of local truths, embodying a multidimensional and multi-parallel reality.
    \par
    \textit{Superpositivism} values the collective wholeness of subcultures without elevating any particular one. It critiques the binary classification of culture into "high" and "low", recognizing that factors like publication and popularity don't necessarily define a work's value.
    \par
    While postmodernism asserts the absence of meaning and nihilism the unnecessary nature of meaning, \textit{superpositivism} promotes a unifying consolidation. Today's cultural milieu yearns for expansive, deep, cohesive narratives and ideas, shunning all-encompassing global narratives (métarécit or grand récit). The fatigue from the metastable instability of 'clip' temporal constructs should dampen this dance of Shiva that postmodernism's destructive aspects so readily parallel.




 
\bibliographystyle{unsrt}
\bibliography{refs}
\end{document} 